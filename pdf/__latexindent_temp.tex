% Project Description

\documentclass[12pt]{article}
\usepackage[letterpaper, margin=1in]{geometry}
\usepackage[utf8]{inputenc}
\usepackage{physics}
\usepackage{amsmath}
\usepackage{amssymb}
\usepackage{hyperref}
\usepackage{listings}

\begin{document}

\begin{center}
    \textbf{\Large Poisson's Spot Simulator}
\end{center}

\textbf{Group Members:}

\qquad Zhexing Zhang

\qquad Renjie Zhu

\section{Installing Required Packages}

The python modules required to run this program are,
\begin{itemize}
    \item \href{http://www.numpy.org/}{numpy}
    \item \href{https://matplotlib.org/}{matplotlib}
    \item \href{https://pypi.org/project/PyQt5/}{PyQt5}.
\end{itemize}

All three modules should have already been installed on the 
Raspberry Pi configured for PHYS 129L, but for use on 
other systems (including Windows 10, macOS and full-fledged
Linux), please run the following commands in the corresponding
command line tools.

{\fontfamily{lmtt}\selectfont 
\begin{lstlisting}
    pip3 install numpy
    pip3 install matplotlib
    pip3 install pyqt5
\end{lstlisting}
}

Due to the ARM architecture of the processor of Raspberry 
Pi 3 Model B, the above commands may not work and user 
should look for other solutions if any modules above are 
not installed. 

\section{External Hardwares}

There are no external hardwares needed for this program to
function.

\section{Program Description}

The program we wrote is meant to simulate Poisson Spot using Huygens–Fresnel principle. 

According to the principle, each point on the light path can be seen as a secondary point light source, so I used every grid point outside the disk to be a light source with same phase. The intensity of a point on the screen will be the superposition of all these points. The phase shifts of the points will cause interference phenomenon on the screen, and finally lead to the appearance of Poisson Spot.

The phase shifts of the light sources are caused by the difference of the distance between the sources and the point on the screen. Following this idea, I calculated the distance between each light source and the point as a matrix, and then used the matrix to calculate the phases of the lights using exponential expression ($e^{i\theta}$, where $\theta = \frac{2 \pi l}{\lambda}$). The absolute value of the sum of the phases is the intensity on that point. 

Because the Poisson Spot is a radial symmetric plot of the interference of light outside the disk, I calculated the intensity of light at different distance to the origin, instead of calculating the intensity for each point on the x-y plane. Next, I used the function of intensity in terms of radius to adjust the intensity in terms of x and y, so that I can save time for repeated calculation on points of same radius. To accomplish this method, after calculating the intensity of one radius, the program will change the intensity value of points inside the radius and repeat the process for each radius from the largest radius to 0. 

Then plot it and we get the final plot showing Poisson Spot.

A faster way suggested by Professor Lipman is using the complement light method. This method states that the intensity of a hole with same radius of the disk is complement to the intensity of the disk. If we add the intensity of them, we will get a constant light intensity on the screen. This is a promising method to speed up the program, but when we wrote the program for the complement light method, we found out that the constant intensity on the screen is a complex number with unknown phase, which ruined the plausibility of the method. 

\section{Results}

To run this program, go to the project directory and run
the following command:
{\fontfamily{lmtt}\selectfont 
\begin{lstlisting}
    $project_directory$ main.py
\end{lstlisting}
}

You will be greeted with an introduction of the program 
when you start (fig. \ref{fig:greet}). Press 
OK to go to the input interface (fig. \ref{fig:input}).
A suggestion is shown on this page, but you can change 
these parameters as long as they meet the requirements
of poisson spot interference pattern. The program will
take about 10 minutes on a Raspberry Pi 3 Model B to run
and show the image. After the image is shown, 
the image will be saved as an eps
file with parameters in the filename to avoid duplication.

After the first image is shown, you can change the parameters
and compute another interference pattern using outher 
parameters. Followed are some example images created by
this program.



\section{Credits}

This program is divided into two main parts: GUI 
and algorithm. Renjie Zhu is responsible mainly for the 
GUI part and other user interactive parts. Zhexing Zhang 
is responsible for implementing the algorithm that 
generates a grid of resulting intensity from the diffraction.
However, extensive collaboration is the key during the
process of writing this program and is the reason this 
program works as described.

We want to thank Jiahao for testing the program 
and giving feedback about the user experience. We also
want to thank Prof. Lipman for the insight and help on the 
problems we were trying to solve.

\end{document}